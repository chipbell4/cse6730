\documentclass[a4paper,12pt]{article}

\begin{document}

\title{Project 1 Checkpoint}
\author{Chip Bell}
\date{February 6th, 2015}
\maketitle

\section{Problem Description}
Traffic simulation is becoming increasing important due to the expansion and growth of cities. Long-standing
infrastructure is forced to adapt in order to accomodate greater amounts of traffic, despite the fact that in many
cases the roads cannot be modified. This can be a result of legislation, existing buildings, and private property among
other things. So as a result, cities must instead optimize the variables they can control. In this case, we consider
signal light timing for the 12th St. and Peachtree Street intersection in only the north and south directions due to
time constraints, and the sheer difficulty with modeling traffic in general. In particular, we explore the effects that
light timing has on traffic flow, given a source input distribution of cars.

\section{The Conceptual Model}
In order to create a representative software simulation of the aforementioned system, we first begin by designing a
conceptual version of this model from which we can plan our implementation. Data from \cite{cts12} was used extensively
for modeling purposes in this project.

\subsection{Simulation Input}

\subsection{Simulation Output}

\subsection{Content}

\subsection{Simplifications and Assumptions}
Probably the most obvious "flaw" in such a design are the simplifications made. Details can certainly be left out
purposely for many reasons such as time constraints, lack of supporting data to model a extra feature, or simply
unneccesary complexity.

In this case, certainly the most prominent would be the ``isolation'' of the model. We are only considering an
idealized random input distribution of cars from both North and South of the 12th St. intersection. However, this is
certainly not the case, since this input distribution is affected by countless other factors like the surrounding
lights, cars passing along 12th at that intersection, and even simply time of day. We assume in this simulation is
perpetually between 4:00PM to 4:15PM in Atlanta (a frightening thought indeed), simply due to the fact that we only
have data for that time range. All in all, we assume the conditions were fairly identical to that in which the 
aforementioned CTS study was made.

\section{Simulation Software Architecture}
Simulation Software

\section{Current Build}
Current Build

\cite{cts12}
\bibliographystyle{plain}
\bibliography{main}

\end{document}
