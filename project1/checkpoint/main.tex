\documentclass[a4paper,12pt]{article}

\usepackage{url, hyperref}

\begin{document}

\title{Project 1 Checkpoint}
\author{Chip Bell}
\date{February 6th, 2015}
\maketitle

\section{Problem Description}
Traffic simulation is becoming increasing important due to the expansion and growth of cities. Long-standing
infrastructure is forced to adapt in order to accomodate greater amounts of traffic, despite the fact that in many
cases the roads cannot be modified. This can be a result of legislation, existing buildings, and private property among
other things. So as a result, cities must instead optimize the variables they can control. In this case, we consider
signal light timing for the 12th St. and Peachtree Street intersection due to
time constraints, and the sheer difficulty with modeling traffic in general between multiple intersections. In particular,
we explore the effects that light timing has on traffic flow, given a source input distribution of cars.

\section{The Conceptual Model}
In order to create a representative software simulation of the aforementioned system, we first begin by designing a
conceptual version of this model from which we can plan our implementation. Data from \cite{cts12} was used extensively % TODO: Put this somewhere else
for modeling purposes in this project. Our architecture will be event driven, prioritizing events by timestamp. Furthermore,
we'll utilize the publisher-subscriber pattern (see \cite{pubsub}) to promote better decoupling.
Given this architecture, our first step will be to identify the entities present in our model, and then consider their
interactions (or the inherent ``activies'') within the simulation. This will allow us to then design the events that
are important

\subsection{Entities}
The first entity, unsurprisingly, is a single car. Our representation of a car is clearly less detailed than the real
thing, but rather focuses on variables that influence it's travel time from its starting location to destination, along
with the variables that influence neighboring cars. For our simulation, we'll track the position and destination of the
cars, along with the size of the car since that can influence the position of cars behind it. Also, we will track the
current velocity of the car 

Another entity to consider is the intersection itself. It's changing in time constantly, and those changes affect
every entity within the simulation. The most important parameters to consider here are the light timings. This
intersection's timings are time-based, rather than sensor-based, which can be seen in the data set provided with the
project description and is also available online \cite{ngsim}. We assume that the current light timing provided here
are optimal, but because we'll have a computer simulation in the end we have free reign in experimenting with these
values and observing the outcome. Perhaps the timings are \emph{not} optimal. Perhaps, our model grossly oversimplifies
some crucial interaction within the system. This is perhaps the most exciting part of the project.

\subsection{Basic Interactions}
From the intersection entity, we can glean an important interaction that occurs: The changing of light signal. When a
signal changes, each car in-queue will be forced to change it's state, whether it's waiting or currently in motion.
From an event-driven perspective, this means that any listener would need to be able to know \emph{how} the light
changed in order to act accordingly.

An intersection serves as holding point for cars traveling along the road so if a light is red, cars are forced to
queue up at the intersection until green. This lends itself to a queue-style data structure that encorporates both
position and acceleration. Car acceleration is a complex topic, but models have been constructed \cite{bonneson}
\cite{herman_et_al}. These models are generally continuous, and can generally be represented as a differential equation
due to the relationships between acceleration and velocity, as can be seen in \cite{briggs} \cite{deceleration} as well.
Instead, we will
simply consider the various events that trigger changes in behavior. For instance, if a car slows down, the car behind
it will be notified and change it's driving patterns accordingly.

\subsection{Events}
Based on our observations from above, we can now construct a set of events that are pertinent to this application.

First off, signal changes consititute the major event in the system. When a signal changes, the event must contain
\emph{when} the signal changed, and \emph{what} the signal changed to. The light itself will listen for this event
and will enqueue the next signal change based on the signal change time. These signal changes could be enqueued
completely upfront, but allowing this to be done on the fly allows the user to tweak these values during the simulation.

The car model will listen for this event as well. When a currently traveling car sees a red light, it will calculate
\emph{where} it needs to stop, and \emph{when} it will come to a stop. This will be done using a deceleration model
similar to \cite{deceleration}. If the light turns to green, the car instead needs to start accelerating (using one of
the models mentioned above). The model will adapt to light changes during movement by recalculating its estimated
exit time of the system.

\subsection{Simulation Input}
The two main sources of input into this simulation are the cars entering the section of Peachtree St with which we are
concerned, and the light timing of the system. The combination of these two variables are what influence the smoothness
of traffic flow to the actual drivers in the system.

We first concern ourselves with finding an input distribution for cars entering the system. In our model, we'll only
consider the car arrival time, and car length when modeling cars. Due to the discrete nature of the application, we'll
ignore some of the more "continuous" aspects, such car velocity and acceleration and encompass them into calculation
of event times.

\subsection{Simulation Output}

\subsection{Simplifications and Assumptions}
Probably the most obvious "flaw" in such a design are the simplifications made. Details can certainly be left out
purposely for many reasons such as time constraints, lack of supporting data to model a extra feature, or simply
unneccesary complexity.

In this case, certainly the most prominent would be the ``isolation'' of the model. We are only considering an
idealized random input distribution of cars from both North and South of the 12th St. intersection. However, this is
certainly not the case, since this input distribution is affected by countless other factors like the surrounding
lights, cars passing along 12th at that intersection, and even simply time of day. We assume in this simulation is
perpetually between 4:00PM to 4:15PM in Atlanta (a frightening thought indeed), simply due to the fact that we only
have data for that time range. All in all, we assume the conditions were fairly identical to that in which the 
aforementioned CTS study was made.

\section{Simulation Software Architecture}
Simulation Software

\section{Current Build}
Current Build

\cite{cts12}
\bibliographystyle{plain}
\bibliography{main}

\end{document}
