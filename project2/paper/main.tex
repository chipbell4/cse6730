\documentclass[a4paper,12pt]{article}

\usepackage{url, hyperref, graphicx}

\begin{document}

\title{Project 2}
\author{Chip Bell}
\date{March 31, 2015}
\maketitle

\section{Problem Description}
The Washington Metropolitan Area Transit Authority's Metrorail system in many respects is a driving factor in the
expansion of the city. With Washington's fixed size, reliable and cost-efficient transportation for metro area
residents is crucial for maintaining the city's economy. However, this is a difficult task in that there are many
variables involved, such as human error, mechanical failure, and even acts of nature.

Therefore for this project, I will be simulating the metrorail system using data collected from the WMATA API
\cite{wmataapi}, which provides realtime estimates of train arrival. Furthermore, I will simulate the impact that track
closures have on train throughput.

\section{Previous Work}
Many large cities have rapid transit systems, so considerable research time has been invested on building reliable
subways. Commercial software such as OpenTrack \cite{opentrack} have been developed, and even video games such as Train
Simulator \cite{trainsimulator} and Open Rails \cite{openrails}. Algorithms for handling single, double and tracks exist, such as the basic intuitive
algorithm for sharing tracks between trains presented by Fiorini and Botter \cite{fioroni}. Dessouky and Leachman
\cite{dessouky_leachman_95} provide an event-driven algorithm for single and double track rail networks, incorporating
acceleration. Lu et al \cite{quan_lu} expand the model to triple tracks while enhancing the acceleration model to
handle multiple track speed limits.

\section{Basic Architecture}

\section{Data Collection}

\section{Caveats and Limitations}

\section{Technology Used}

\section{Remaining Work}

\bibliographystyle{plain}
\bibliography{main}

\end{document}
